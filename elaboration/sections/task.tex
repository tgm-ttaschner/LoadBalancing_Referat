%!TEX root=../document.tex

\section{Einführung}
Diese Übung soll zur Vertiefung der Begriffe "'Authentifizierung und Autorisierung"' dienen.

\subsection{Ziele}
Das Ziel dieser Übung ist die Funktionsweise eines Verzeichnisdienstes zu verstehen und Erfahrungen mit der Administration auszuprobieren. Ebenso soll die Verwendung des Dienstes aus einer Anwendung heraus mit Hilfe der JNDI geübt werden.

Authentifizierung bedeutet hier, dass per Username und Passwort eine Anmeldung beim Verzeichnisdienst erfolgt. Autorisierung wird hier im Zusammenhang mit Service-Gruppen und zugeordneten Usern durchgeführt.


\subsection{Voraussetzungen}
\begin{itemize}
\item Grundlagen Verzeichnisdienst
\item Administration eines LDAP Dienstes
\item Verwendung von Commandline Werkzeugen für LDAP (LDAPSEARCH, LDAPMODIFY)
\item Grundlagen der JNDI API für eine JAVA Implementierung
\item Verwendung einer virtuellen Instanz für den Betrieb des Verzeichnisdienstes
\end{itemize}


\subsection{Aufgabenstellung}
Mit Hilfe der zur Verfügung gestellten VM wird ein vorkonfiguriertes LDAP Service zur Verfügung gestellt. Dieser Verzeichnisdienst soll um folgende Einträge erweitert werden. Das verwendete Namensschema (eg. group.service1 oder vorname.nachname) soll für alle Einträge verwendet werden.

\begin{itemize}
\item 5 Posix Groups (beliebe Zuweisung von UserIDs)
\item 10 User Accounts
\end{itemize}

Weiters soll eine Java-Applikationen zur Authentifizierung und Autorisierung entwickelt werden. Folgende Fragestellungen stehen dabei im Mittelpunkt:

\begin{itemize}
\item Sind Username und Passwort korrekt?
(Identifikation des Benutzers)
\item Ist der User berechtigt ein bestimmtes Service zu nutzen?
(Benutzer-Berechtigung)
\end{itemize}
\newpage

\uline{Bewertung}: 16 Punkte

\begin{list}{-}
\item Dokumentation der einzelnen Arbeitsschritte im Protokoll (2 Punkte)
\item Anlegen von 5 Gruppen und 10 User Accounts (6 Punkte)
(wenn fremdes LDAP-Service verwendet wird, dann Dokumentation von 3 LDAPSEARCH und 2 LDAPMODIFY Befehlen)
\item Authentifizierung (4 Punkte)
\item Autorisierung (4 Punkte)
\item Wie ist eine LDAP Änderung moeglich mit bestimmten Benutzer (ungleich admin)?
\item Brute Force Implementierung
\end{list}

\clearpage
