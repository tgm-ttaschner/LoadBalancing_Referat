%!TEX root=../document.tex

\section{Ergebnisse}
\label{sec:Ergebnisse}

\subsection{Einrichten der VM}
\label{sec:Einrichten der VM}

\subsubsection{Installation von LDAP}
\label{sec:Installation von LDAP}

\begin{lstlisting}[frame=single, caption=Installation von LDAP]
sudo apt-get update
sudo apt-get install slapd ldap-utils
\end{lstlisting}

Nun erfolgt die Konfiguration des Pakets slapd. Hierzu muss folgendes bei der Installation eingestellt werden:
\begin{lstlisting}[frame=single, caption=Konfiguration von slapd]
DNS domain name: nodomain.com
Organization name: nodomain
Administrator password: user
Database backend: hdb
\end{lstlisting}

\subsubsection{Installation von phpLDAPadmin}
\label{sec:Installation von phpLDAPadmin}
Die Installation von phpLDAPadmin lässt sich mit dem folgenden Befehl bewerkstelligen:
\begin{lstlisting}[frame=single, caption=Installation von phpLDAPadmin]
sudo apt-get install phpldapadmin
\end{lstlisting}

\subsubsection{Konfiguration von phpLDAPadmin}
\label{sec:Konfiguration von phpLDAPadmin}
In diesem Schritt wird die Datei /etc/phpldapadmin/config.php um folgende Zeilen ergänzt:
\begin{lstlisting}[frame=single, caption=Konfiguration von phpLDAPadmin]
$servers->setValue('server','host','localhost');
$servers->setValue('server','base',array('dc=nodomain,dc=com'));
$servers->setValue('login','bind_id','cn=admin,dc=nodomain,dc=com');
$config->custom->appearance['hide_template_warning'] = true;
\end{lstlisting}

Eine Konfiguration für eine SSL gesicherte Verbindung wird hier nicht vorgenommen.

\subsubsection{Konfiguration von Apache}
\label{sec:Konfiguration von Apache}
Abschließend muss nur noch ein Alias-Eintrag in der Datei /etc/apache2/mods-enabled/alias.conf erfolgen.
\begin{lstlisting}[frame=single, caption=Hinzufügen eines Alias-Eintrags]
<IfModule alias_module>
	...
	Alias /ldap /usr/share/phpldapadmin/htdocs
	...
</IfModule>
\end{lstlisting}
Wichtig dabei ist, dass sich der Eintrag innerhalb des IfModule Tags befindet.
PhpLDAPAdmin sollte nun unter http://localhost/ldap erreichbar sein.

\subsection{Anlegen von 5 Gruppen und 10 Personen}
\label{sec:Anlegen von 5 Gruppen und 10 Personen}

Zunächst muss ein erfolgreicher Login als admin erfolgen, die Startseite sollte angezeigt werden.

\begin{figure}[!h]
	\begin{center}
		\includegraphics[width=1.0\linewidth]{images/ldapadmin_main.png}
		\caption{phpLDAPadmin Startseite}
		\label{ldapadmin_main}
	\end{center}
\end{figure}

Im Frame My LDAP Server muss auf der linken Seite nun der Verzeichnisbaum aufgeklappt werden.
\newpage

\subsubsection{Erstellen einer Gruppe}
\label{sec:Erstellen einer Gruppe}
Hierzu erfolgt erfolgt ein Klick auf den Menüeintrag "'Neuen Eintrag erzeugen"'. Im rechten Frame muss nun eine Vorlage für das neu zu erstellende Objekt ausgewählt werden. Für eine Gruppe wird die Vorlage "'Allgemein: POSIX-Gruppe"' ausgewählt.

Für das Erzeugen eines neuen Eintrages muss lediglich nur der Gruppenname eingetragen werden. 
Es wurde mit dem Betreuer eine bestimme Namensgebung der Gruppen ausgemacht. Jede Gruppe muss den Namen group.service[n] tragen, wobei n eine Zahl zwischen 1 und 5 ist.
Die Gruppen-ID wird selbstständig vergeben. Optional können auch bereits vorhandene Benutzer der Gruppe hinzugefügt werden. Abschließend kann der Eintrag per Knopfdruck abgespeichert werden.

\begin{figure}[!h]
	\begin{center}
		\includegraphics[width=0.5\linewidth]{images/ldapadmin_addgroup.png}
		\caption{Erstellen einer POSIX-Gruppe}
		\label{ldapadmin_addgroup}
	\end{center}
\end{figure}

\textbf{Anmerkung:} Im Laufe der Übungsstunde bekamen wir von unserem Betreuer den Tipp die Gruppen zuerst anzulegen, da die Benutzer nachher beim Erstellen einer Gruppe zugewiesen werden müssen.

\newpage

\subsubsection{Erstellen einer Person}
\label{sec:Erstellen einer Person}
%%Für die Erstellung der Gruppen habe ich phpLDAPadmin verwendet. Dafür habe ich mich als Admin angemeldet, damit ich keine Einschränkungen bei den Rechten habe. In der Drop-Down Liste "'dc=nodomain, dc=com"', welche auf der linken Seite ist, gibt es einen Button "'Neuen Eintrag erzeugen"'. Nachdem man diesen geklickt hat gibt es mehrere Vorlagen zur Auswahl. Von diesen Vorlagen habe ich Benutzerkonto ausgewählt. Nun musste man nur mehr ein simples Formular ausfüllen. Von Name bis hin zum Passwort.
Auch hier erfolgt erfolgt ein Klick auf den Menüeintrag "'Neuen Eintrag erzeugen"'. Im rechten Frame muss nun eine Vorlage für das neu zu erstellende Objekt ausgewählt werden. Für ein Benutzerkonto wird die Vorlage "'Allgemein: Benutzerkonto"' ausgewählt.

Für das Erzeugen eines neuen Eintrages müssen der Benutzername (hier als Üblicher Name dargestellt), die Gruppenzuweisung, das Heimverzeichnis, der Nachname und die Benutzer-ID eingetragen werden. Auch hier wurde mit dem Betreuer ausgemacht, dass der Benutzername des Benutzers im Format vorname.nachname eingetragen wird. Optional können noch ein Vorname, eine Login shell und ein Passwort festgelegt werden. Die Benutzer-ID in numerischer Form wird automatisch generiert.
Abschließend kann der Eintrag per Knopfdruck abgespeichert werden.

\begin{figure}[!h]
	\begin{center}
		\includegraphics[width=0.4\linewidth]{images/ldapadmin_adduser.png}
		\caption{Erstellen eines Benutzerkontos}
		\label{ldapadmin_adduser}
	\end{center}
\end{figure}

\textbf{Anmerkung:} Der Benutzername kann automatisch anhand des Vor- und Nachnamen generiert werden. Es wäre von Vorteil beide Werte vorher einzutragen.
\newpage

\subsubsection{Gruppenzuweisungen}
\label{sec:Gruppenzuweisungen}

Um der Aufgabenstellung gerecht zu werden, wurden 5 Gruppen und 10 Benutzer angelegt. Welcher Benutzer welcher Gruppe zugeteilt wurde, kann der folgenden Tabelle entnommen werden.

\renewcommand{\arraystretch}{1.5}
\begin{table}[!h]
	\center
	\begin{tabular}{
		|
		@{\hspace{5mm}} c @{\hspace{5mm}} |
		@{\hspace{5mm}} c @{\hspace{5mm}} |
		@{\hspace{5mm}} c @{\hspace{5mm}} |
		@{\hspace{5mm}} c @{\hspace{5mm}} |
		@{\hspace{5mm}} c @{\hspace{5mm}} |
		@{\hspace{5mm}} c @{\hspace{5mm}} |
	}
		\hline
		
		& \shortstack[c]{
			\textbf{service.} \\ 
			\textbf{group1}
		}
		& \shortstack[c]{
			\textbf{service.} \\ 
			\textbf{group2}
		}
		& \shortstack[c]{
			\textbf{service.} \\ 
			\textbf{group3}
		}
		& \shortstack[c]{
			\textbf{service.} \\ 
			\textbf{group4}
		}
		& \shortstack[c]{
			\textbf{service.} \\ 
			\textbf{group5}
		}
		\\
		\hline
		
		\textbf{patrick.malik}
		& 
		& 
		& 
		& X
		& 
		\\
		\hline		
		
		\textbf{thomas.micheler}
		& 
		& X
		& 
		& 
		& 
		\\
		\hline
		
		\textbf{tobias.perny}
		& 
		& 
		& 
		& X
		& 
		\\
		\hline
		
		\textbf{stefan.polydor}
		& X
		& 
		& 
		& 
		& 
		\\
		\hline
		
		\textbf{manuel.reilaender}
		& 
		& 
		& X
		& 
		& 
		\\
		\hline
		
		\textbf{matthias.ritter}
		& 
		& 
		& 
		& 
		& X
		\\
		\hline
		
		\textbf{sebastian.steinkellner}
		& 
		& 
		& X
		& 
		& 
		\\
		\hline
		
		\textbf{thomas.taschner}
		& X
		& 
		& 
		& 
		& 
		\\
		\hline
		
		\textbf{michael.weinberger}
		& 
		& X
		& 
		& 
		& 
		\\
		\hline
		
		\textbf{simon.wortha}
		& 
		& 
		& 
		& 
		& X
		\\
		\hline
	\end{tabular}
	\caption{Gruppenzuweisungen der Benutzer}
	\label{methoden}
\end{table}

Der Benutzer muss der Gruppe hinzugefügt werden (memberUid setzen)!

\subsection{LDAPSEARCH}
\label{sec:LDAPSEARCH}

\subsubsection{Erklärung des Befehls}
\label{sec:Erklärung des Befehls}

\begin{lstlisting}[frame=single, caption=LDAPSEARCH Befehl]
ldapsearch -h 192.168.0.8 -p 389 -D "cn=max.mustermann,dc=nodomain,dc=com" -W -b "dc=nodomain,dc=com"
\end{lstlisting}

Zur Suche sind die folgenden Parameter relevant:
\\

-h... Adresse des Servers, auf dem der LDAP-Dienst ausgeführt wird

-p... Port, auf dem der LDAP-Dienst erreichbar ist

-D... Anmeldedaten für den LDAP-Dienst (distinguished name)

-W... Passwort

-b... Startpunkt für die Suche
\cite{LdapFlags}
\newpage

\subsubsection{LDAPSEARCH 1}
\label{sec:LDAPSEARCH 1}

\begin{lstlisting}[frame=single, caption=LDAPSEARCH 1]
ldapsearch -h 127.0.0.1 -p 389 -D "cn=admin,dc=nodomain,dc=com" -W
Enter LDAP Password: 
# extended LDIF
#
# LDAPv3
# base <> (default) with scope subtree
# filter: (objectclass=*)
# requesting: ALL
#

# search result
search: 2
result: 32 No such object

# numResponses: 1
\end{lstlisting}


Es wurde lediglich nur ein Bind auf unsere lokale Instanz durchgeführt.

\subsubsection{LDAPSEARCH 2}
\label{sec:LDAPSEARCH 2}

\begin{lstlisting}[frame=single, caption=LDAPSEARCH 2]
ldapsearch -h 192.168.188.34 -p 389 -D "cn=admin,dc=nodomain,dc=com" -W
Enter LDAP Password: 
# extended LDIF
#
# LDAPv3
# base <> (default) with scope subtree
# filter: (objectclass=*)
# requesting: ALL
#

# search result
search: 2
result: 32 No such object

# numResponses: 1
\end{lstlisting}

Es wurde lediglich nur ein Bind auf unsere entfernte Instanz durchgeführt.
\newpage

\subsubsection{LDAPSEARCH 3}
\label{sec:LDAPSEARCH 3}

\begin{lstlisting}[frame=single, caption=LDAPSEARCH 3]
ldapsearch -h 192.168.188.34 -p 389 -D "cn=thomas.taschner,dc=nodomain,dc=com" -W -b "dc=nodomain,dc=com"
Enter LDAP Password: 
# extended LDIF
#
# LDAPv3
# base <dc=nodomain,dc=com> with scope subtree
# filter: (objectclass=*)
# requesting: ALL
#

# nodomain.com
dn: dc=nodomain,dc=com
objectClass: top
objectClass: dcObject
objectClass: organization
o: nodomain
dc: nodomain

# admin, nodomain.com
dn: cn=admin,dc=nodomain,dc=com
objectClass: simpleSecurityObject
objectClass: organizationalRole
cn: admin
description: LDAP administrator

# group.service1, nodomain.com
dn: cn=group.service1,dc=nodomain,dc=com
gidNumber: 500
cn: group.service1
objectClass: posixGroup
objectClass: top

...

# search result
search: 2
result: 0 Success

# numResponses: 18
# numEntries: 17

\end{lstlisting}


Eine externe Suche, die uns sämtliche Einträge der Domäne ausgibt.

\subsection{LDAPMODIFY}
\label{sec:LDAPMODIFY}

\subsubsection{LDAPMODIFY 1}
\label{sec:LDAPMODIFY 1}

\begin{lstlisting}[frame=single, caption=LDAPMODIFY 1]
ldapmodify -h 192.168.188.34 -p 389 -D "cn=admin,dc=nodomain,dc=com" -W
Enter LDAP Password: 
dn: cn=group.service1,dc=nodomain,dc=com
changetype: modify
replace: description
description: test

modifying entry "cn=group.service1,dc=nodomain,dc=com"

\end{lstlisting}

Ändert die Beschreibung der Gruppe group.service1 auf test.
Nun wird überprüft, ob die Änderung übernommen wurde.

\begin{lstlisting}[frame=single, caption=LDAPMODIFY 1 Check]
ldapsearch -h 192.168.188.34 -p 389 -D "cn=admin,dc=nodomain,dc=com" -W -b "cn=group.service1,dc=nodomain,dc=com"
...
# group.service1, nodomain.com
dn: cn=group.service1,dc=nodomain,dc=com
gidNumber: 500
cn: group.service1
objectClass: posixGroup
objectClass: top
description: test
...

\end{lstlisting}

\subsubsection{LDAPMODIFY 2}
\label{sec:LDAPMODIFY 2}

\begin{lstlisting}[frame=single, caption=LDAPMODIFY 2]
ldapmodify -h 192.168.188.34 -p 389 -D "cn=admin,dc=nodomain,dc=com" -W
Enter LDAP Password: 
dn: cn=thomas.taschner,dc=nodomain,dc=com
changetype: modify
add: telephoneNumber
telephoneNumber: 0123456789

modifying entry "cn=thomas.taschner,dc=nodomain,dc=com"

\end{lstlisting}

Fügt dem Benutzer thomas.taschner das Attribut telephoneNumber hinzu.
Nun wird überprüft, ob die Änderung übernommen wurde.

\begin{lstlisting}[frame=single, caption=LDAPMODIFY 2 Check]
ldapsearch -h 192.168.188.34 -p 389 -D "cn=admin,dc=nodomain,dc=com" -W -b "cn=thomas.taschner,dc=nodomain,dc=com"
...
# thomas.taschner, nodomain.com
dn: cn=thomas.taschner,dc=nodomain,dc=com
givenName: Thomas
gidNumber: 500
homeDirectory: /home/users/ttaschner
sn: Taschner
loginShell: /bin/sh
objectClass: inetOrgPerson
objectClass: posixAccount
objectClass: top
uidNumber: 1000
uid: ttaschner
cn: thomas.taschner
userPassword:: e01ENX1mcTdNOVlEYUVHd01RL01Ja1VGcFZnPT0=
telephoneNumber: 0123456789
...

\end{lstlisting}
\newpage


\subsection{Authentifizierung}
\label{sec:Authentifizierung}
Hierzu wurde der Code \cite{StackoverflowAuthenticate} übernommen und entsprechend an unsere Bedürfnisse angepasst. Sollte die Autorisierung klappen, so wird ein OK zurückgegeben. Sollte sie fehlschlagen, so wird ein NOK zurückgegeben. Zu Testzwecken werden beide Szenarien ausprobiert.

\begin{lstlisting}[frame=single, language=java, caption=Java Code zur LDAP Authentifizierung]
package ldap;

import java.util.Hashtable;

import javax.naming.*;
import javax.naming.directory.*;

public class LDAPAuthentication {

	private DirContext ctx;
	private String username;
	private final String ldapSearchBase = "dc=nodomain,dc=com";

	public LDAPAuthentication(String host, String username, String pw, int port) {
		this.username = "cn=" + username + ",dc=nodomain,dc=com";
		Hashtable<String, String> env = new Hashtable<String, String>();
		env.put(Context.INITIAL_CONTEXT_FACTORY,
				"com.sun.jndi.ldap.LdapCtxFactory");
		env.put(Context.PROVIDER_URL, "ldap://" + host + ":" + port);

		env.put(Context.SECURITY_AUTHENTICATION, "simple");
		env.put(Context.SECURITY_PRINCIPAL, this.username);
		env.put(Context.SECURITY_CREDENTIALS, pw);

		try {
			ctx = new InitialDirContext(env);
			System.out.println("Authentifizierung: OK");
		} catch (NamingException ne) {
			System.out.println("Authentifizierung: NOK");
			System.exit(1);
		} catch (Exception e) {
			System.exit(1);
		}
	}

	public static void main(String[] args) {
		new LDAPAuthentication("192.168.188.34", "thomas.taschner", "thomas.taschner", 389);
		new LDAPAuthentication("192.168.188.34", "thomas.taschner", "passwort123", 389);
	}

}
\end{lstlisting}

\subsection{Autorisierung}
\label{sec:Autorisierung}
Hier wurde der Code angepasst und erweitert. Übernommen wurde er von hier
\cite{StackoverflowAuthenticate}
\cite{LDAP-CLT}
.
Es wird überprüft, ob der Benutzer Mitglied einer bestimmten Gruppe und daher berechtigt ist den Service zu nutzen. SearchControls ermöglicht es einem, in einem LDAP-Verzeichnis zu suchen. Sollte die Gruppe das Attribut memberuid enthalten, so wird OK ausgegeben. Sollte dies nicht der Fall sein, so wird NOK ausgegeben.

\begin{lstlisting}[frame=single, language=java, caption=Java Code zur LDAP Autorisierung]
package ldap;

import javax.naming.Context;
import javax.naming.NamingEnumeration;
import javax.naming.NamingException;
import javax.naming.directory.*;
import java.util.Hashtable;


public class LDAPAuthorization {

	private DirContext ctx;
	private String username;
	private String ldapSearchBase = "dc=nodomain,dc=com";

	public LDAPAuthorization(String host, String username, String password, int port, String group) {
		this.username = "cn=" + username + ",dc=nodomain,dc=com";
		Hashtable env = new Hashtable();
		env.put(Context.INITIAL_CONTEXT_FACTORY,
				"com.sun.jndi.ldap.LdapCtxFactory");
		env.put(Context.PROVIDER_URL, "ldap://" + host + ":" + port);

		env.put(Context.SECURITY_AUTHENTICATION, "simple");
		env.put(Context.SECURITY_PRINCIPAL, this.username);
		env.put(Context.SECURITY_CREDENTIALS, password);

		try {
			ctx = new InitialDirContext(env);
			System.out.println("Authentifizierung: OK");
		} catch (NamingException ne) {
			System.out.println("Authentifizierung: NOK");
			System.exit(1);
		}

		try {
			SearchControls searchControls = new SearchControls();
			searchControls.setSearchScope(SearchControls.SUBTREE_SCOPE);
			searchControls.setTimeLimit(30000);

			// Ueberpruefung ob user in der angegeben Gruppe ist
			NamingEnumeration<?> namingEnum = ctx.search("cn=" + group + ",dc=nodomain,dc=com", "(objectclass=posixGroup)", searchControls);

			while (namingEnum.hasMore ()) {
				SearchResult result = (SearchResult) namingEnum.next ();
				Attributes attrs = result.getAttributes();
				System.out.println("Authorization: " + group + " " + (attrs.get("memberUID") != null &&
						attrs.get("memberUid").contains(username) ? "OK" : "NOK"));
			}

		} catch (NamingException ne) {
			System.out.println("NOK - Authorization");
		}
	}

	public static void main(String[] args) throws NamingException {
		new LDAPAuthorization("10.0.104.73", "thomas.taschner", "thomas.taschner", 389, "group.service1");
	}
}
\end{lstlisting}

\subsection{Änderungen mit bestimmtem User}
\label{sec:Änderungen mit bestimmtem User}
Dies lässt sich mit Hilfe einer Access Control List umsetzen. Es lassen sich bestimmte Rechte für Benutzer und Gruppen zuteilen.